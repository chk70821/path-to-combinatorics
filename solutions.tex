\documentclass{book}

\usepackage{enumitem}
\usepackage{amsmath}
\usepackage[backref,pdfpagemode=FullScreen,colorlinks=true]{hyperref}

\begin{document}

\title{Solutions to \\
\emph{A Path to Combinatorics for Undergraduates} \\
by Titu Andreescu and Zuming Feng~\cite{andreescu2004path}}
\author{\href{http://profiles.google.com/quantumelixir}{quantumelixir}}
\maketitle

\chapter*{small note}
I am writing this solution manual mostly to motivate myself to finish
all the problems from this book. As a useful side-effect, I also hope
that it would be of use to people who are either stuck on a problem or
want to check their answers against someone else's. However, I am doing
no great good, and my selfish motives take precedence of course. I make
no claims of correctness and shamelessly admit that there might well be
serious mistakes in the answer or in the solution to a problem. I
encourage you to send me corrections by sending pull-requests through
\href{http://github.com/quantumelixir/path-to-combinatorics}{github}.
If you are not well-versed with such marvels of the 21st century, you
can still send me an \href{mailto:quantumelixir@gmail.com}{email}
informing me of the error.

\vspace*{\fill}
\begin{center}
\href{http://creativecommons.org/licenses/by-nc-sa/3.0/}{by-nc-sa 3.0}
\end{center}

\newpage
\vspace*{\fill}
\begin{flushright}
\emph{If people do not believe that mathematics is simple, it is only
because they do not realize how complicated life is.}\\
--John von Neumann
\end{flushright}

\tableofcontents

\vspace*{\fill}
\emph{Combinatorics is an honest subject. No ad\`eles, no sigma-algebras.
You count balls in a box, and you either have the right number or you
haven't. You get the feeling that the result you have discovered is
forever, because it’s concrete. Other branches of mathematics are not so
clear-cut. Functional analysis of infinite-dimensional spaces is never
fully convincing; you don't get a feeling of having done an honest day's
work. Don't get the wrong idea – combinatorics is not just putting balls
into boxes. Counting finite sets can be a highbrow undertaking, with
sophisticated techniques.}\\
\begin{flushright}
--Gian-Carlo Rota
\end{flushright}
\vspace*{\fill}

\chapter{Addition or Multiplication}

\begin{enumerate}[label={1.\arabic*}]

\item
The total number of functions from $\{1, 2, \dots, 1998\} \to \{2000, 2001,
2002, 2003\}$ is $4^{1998}$. For each such function $f$, the sum
$f(1)+f(2)+\dots+f(1998)$ might be either odd or even. If the sum is
even, then $f(1999)$ can be assigned to either of $\{2001, 2003\}$, and if
odd, then $f(1999)$ can be assigned to either of $\{2000, 2002\}$. Hence,
there are two possibilities in each case and required number of
functions is $2\times4^{1998}=2^{3997}$

\item
Let $ab=10a+b$ be the two-digit positive number. For divisibility by
each of it's digits, we have the divisbility conditions: $a | b$, and
$b | 10a$. Taking $b=q_1a$ and $10a=q_2b$, we have $q_1q_2=10$. The only
integer solutions $(q_1,q_2)$ for which are:
$(1,10),(2,5),(5,2),(10,1)$. For each solution, applying the range
constraints $0<a,b<10$, we get the final answer $9 + 4 + 1 + 0 = 14$

\item
We have a bijection between the number of paths from the letter C in
the first row to each R in the last row, and the number of ways to
form the word COMPUTER. With the row index as $n$ and the column index
as $r$, we count the number of paths from the first row to the $(n,
r)$-th element using the recursion ${n \choose r} = {{n-1} \choose {r-1}}
+ {{n-1} \choose r}$. Hence, coefficients in the last row are exactly the binomial
coefficients and we have the answer ${7 \choose 0} + {7 \choose 1} + \dots +
{7 \choose 7} = 2^7 = 128$

\item

\item
The number of ways to order $n$ distinct objects around a circle is
$(n-1)!$. Within each pair, the twins can be permuted in two ways. Thus,
the total number of ways to arrange the twins is $(12-1)! \times 2^{12}$

\item
Separating the $10$ girls by $10$ blank placeholders, we can choose $4$
places for the boys in $10 \choose 4$ ways. Permuting the girls and
boys, after choosing where the boys sit, we get a total of
${10\choose4}10!4!$ ways.

\item
Let the number of black and white marbles in the first and second box be
$b_1, w_1$ and, $b_2, w_2$ respectively. We are given that $b_1 + w_1 +
b_2 + w_2 = 25$. $P(\text{both marbles are black}) = \frac{b_1}{b_1+w_1}
\times \frac{b_2}{b_2+w_2} = \frac{27}{50}$. So have the following
condition: $(b_1+w_1)+(b_2+w_2)=25$ and $(b_1+w_1)(b_2+w_2)|50$, to
which the only solutions $(b_1+w_1,b_2+w_2)$ are $(5,20), (10,15)$
(discarding permutations). When $(b_1+w_1,b_2+w_2)=(5,20)$,
$b_1b_2=54=2\cdot3^3$ and the only possible solution to the number of
black balls in the first and second box is $(3,2\cdot3^2)$. This yields
$\frac{w_1w_2}{(b_1+w_1)(b_2+w_2)}=\frac{(5-3)\cdot(20-2\cdot3^2)}{5\cdot20}=\frac{1}{25}$.
When $(b_1+w_1,b_2+w_2)=(l0,15)$, $b_1b_2=81=3^4$ and the only possible
solution to the number of black balls in the first and second box is
$(3^2,3^2)$. This yields $\frac{w_1w_2}{(b_1+w_1)(b_2+w_2)} =
\frac{(10-3^2)\cdot(15-3^2)}{10\cdot15} = \frac{1}{25}$. Thus, in each
case, the answer is $\frac{1}{25}$


\item

\item
Each element of $S=\{1, 2, \dots, 2003\}$ goes to one set from the ordered
triple. Thus, the total number of ways of assigning all the elements
from $S$ is $3^{2003}$.

\item
The question is equivalent to finding the number of increasing
arithmetic progressions in the set $S=\{1, 2, \dots, 2000\}$.  Suppose
we choose $1$ as the first term. It is clear that we can only choose the
odd numbers $\{3, 5, \dots, 1999\}$ for the middle term to be an integer
and lie in $S$.  There are exactly $1000$ such arithmetic progressions
beginning with $1$.  We argue similarly when the first term is $2, 3,
\dots, 1998$. Thus, the total number of ways to select a three term
increasing arithmetic progression from $S$ is $\underbrace{1000}_{1} +
\underbrace{1000}_{2} + \underbrace{999}_{3} + \underbrace{999}_{4} +
\dots + \underbrace{2}_{1997} + \underbrace{2}_{1998} =
2\times\left(\frac{(1000)(1001)}{2}-1\right) = 1000\times1001-2 =
1000998$.

\item

\item
\end{enumerate}

\chapter{Combinations}
\chapter{Properties of Binomial Coefficients}
\chapter{Bijections}
\chapter{Recursions}
\chapter{Inclusion and Exclusion}
\chapter{Calculating in Two Ways: Fubini's Principle}
\chapter{Generating Functions}
\chapter{Review Exercises}

\bibliographystyle{plain}
\bibliography{references}

\newpage
\begin{flushright}
\vspace*{\fill}
\emph{God save you if you still can't count.}\\
--Anonymous
\end{flushright}

\end{document}
